\documentclass[conf]{new-aiaa}
%\documentclass[journal]{new-aiaa} for journal papers
\usepackage[utf8]{inputenc}

\usepackage{graphicx}
\usepackage{amsmath}
\usepackage{commath}
\usepackage[version=4]{mhchem}
\usepackage{siunitx}
\usepackage{longtable,tabularx}
\usepackage{float}
\usepackage{listings}
\usepackage{pdfpages}
\usepackage{url}
\usepackage{color} %red, green, blue, yellow, cyan, magenta, black, white
\definecolor{mygreen}{RGB}{28,172,0} % color values Red, Green, Blue
\definecolor{mylilas}{RGB}{170,55,241}
\setlength\LTleft{0pt} 

\lstset{language=Matlab,%
	basicstyle=\footnotesize,
	breaklines=true,%
	morekeywords={matlab2tikz},
	keywordstyle=\color{blue},%
	morekeywords=[2]{1}, keywordstyle=[2]{\color{black}},
	identifierstyle=\color{black},%
	stringstyle=\color{mylilas},
	commentstyle=\color{mygreen},%
	showstringspaces=false,%without this there will be a symbol in the places where there is a space
	numbers=left,%
	numberstyle={\tiny \color{black}},% size of the numbers
	numbersep=9pt, % this defines how far the numbers are from the text
	emph=[1]{for,end,break},emphstyle=[1]\color{red}, %some words to emphasise
	%emph=[2]{word1,word2}, emphstyle=[2]{style},    
}

% ================================================================ % 
\title{ASE387P.2 Mission Analysis and Design \\ Homework 5: TSX/TDX Orbit Design}

\author{Junette Hsin}
\affil{Masters Student, Aerospace Engineering and Engineering Mechanics, University of Texas, Austin, TX 78712}

\begin{document}

\maketitle

% \begin{abstract}

	% Theory and algorithms 

% \end{abstract}

% \newpage 
% ================================================================ % 
\section*{Problem 1: }



% \newpage 
% ================================================================ % 
\section*{Problem 2: }



% ================================================================ % 
% Problem 3 

\section*{Problem 3: D'Amico vs. Lim frozen ground-track repeat orbit}




% \newpage
% ================================================================ % 
\section*{Appendix} 

\subsection*{MATLAB code} 

In D'Amico's paper, the driving requirements for the TS-X orbit are: 

\begin{itemize}
	\item exact 11 day repeat cycle for ground track 
	\item sun-synchronicity 
	\item frozen-orbit at about 500 km altitude
	\item mean local time of 18 h ar the ascending node  
\end{itemize}

The value selected for the draconic period, P, is 11/167 days (repetion cycle of 11 days, 167 orbits in the repeat). 

If considering only two-body potential ($J_0$ and $J_1$), the period and semi-major axis, $a_{J_1}$ of an elliptical orbit are related through: 

\begin{equation}
	a_{J_1} = \Bigg( \frac{P}{2 \pi} \sqrt{ GM _{\oplus}  } \Bigg)^{\frac{2}{3}}
	\label{eq:a_J1}
\end{equation}

Expanding geo-potential to include $J_2$ term and neglecting eccentricity: 

\begin{equation}
	a_{J_2} = a_{J_1} + \frac{1}{J_2 GM _{\oplus}} \Bigg( \frac{4 \dot{\Omega} a^3_{J_1}}{ 3 R_{\oplus} } \Bigg)^2 - \frac{J_2 R_{\oplus}^2 }{ a_{J_1} }
\end{equation}

The regression of the right ascension of the ascending node is imposed by the sun-synchronicity requirement, from which we can obtain $i_{J_2}$: 

\begin{equation}
	\dot{\Omega} = \frac{2 \pi}{year} = - \frac{3}{2} \sqrt{GM_{\oplus}} J_2 \frac{R_{\oplus}^2}{a_{J_2}^{3.5}} 
\end{equation}

\begin{equation}
	i_{J_2} = arccos \Bigg( -\frac{2}{3} \frac{\dot{\Omega} a_{J_2}^{3.5} }{ \sqrt{ GM_{\oplus} J_2 R_{\oplus}^2 } } \Bigg)
\end{equation}



% ================================================================ % 

% \bibliography{sample}

\end{document}
