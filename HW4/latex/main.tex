\documentclass[conf]{new-aiaa}
%\documentclass[journal]{new-aiaa} for journal papers
\usepackage[utf8]{inputenc}

\usepackage{graphicx}
\usepackage{amsmath}
\usepackage{commath}
\usepackage[version=4]{mhchem}
\usepackage{siunitx}
\usepackage{longtable,tabularx}
\usepackage{float}
\usepackage{listings}
\usepackage{pdfpages}
\usepackage{url}
\usepackage{color} %red, green, blue, yellow, cyan, magenta, black, white
\definecolor{mygreen}{RGB}{28,172,0} % color values Red, Green, Blue
\definecolor{mylilas}{RGB}{170,55,241}
\setlength\LTleft{0pt} 

\lstset{language=Matlab,%
	basicstyle=\footnotesize,
	breaklines=true,%
	morekeywords={matlab2tikz},
	keywordstyle=\color{blue},%
	morekeywords=[2]{1}, keywordstyle=[2]{\color{black}},
	identifierstyle=\color{black},%
	stringstyle=\color{mylilas},
	commentstyle=\color{mygreen},%
	showstringspaces=false,%without this there will be a symbol in the places where there is a space
	numbers=left,%
	numberstyle={\tiny \color{black}},% size of the numbers
	numbersep=9pt, % this defines how far the numbers are from the text
	emph=[1]{for,end,break},emphstyle=[1]\color{red}, %some words to emphasise
	%emph=[2]{word1,word2}, emphstyle=[2]{style},    
}

% ================================================================ % 
\title{ASE387P.2 Mission Analysis and Design \\ Homework 4: Orbit Design}

\author{Junette Hsin}
\affil{Masters Student, Aerospace Engineering and Engineering Mechanics, University of Texas, Austin, TX 78712}

\begin{document}

\maketitle

% \begin{abstract}

	% Theory and algorithms 

% \end{abstract}

% \newpage 
% ================================================================ % 
\section*{Problem 1: ISS Ground Tracks}



% ------------------------- % 
\subsection*{A}

The following parameters were used to initialize the ISS orbit, which were taken from \url{https://in-the-sky.org/spacecraft_elements.php?id=25544} on April 19: 

% e0 = 0.00049; 
% i0 = 51.6427 * pi/180; 
% w0 = 40.1116 * pi/180; 
% O0 = 0; 
% M0 = 70.88 * pi/180; 
\begin{itemize}
	\item $a$ = 6794.588 km 
	\item $e$ = 0.00049 
	\item $i$ = 51.627 deg 
	\item $\omega$ = 40.1116 deg 
	\item $\Omega$ = 0 deg 
	\item $M$ = 70.88 
\end{itemize}

The right ascension of the ascending node, $\Omega$, was set to 0 as stated in the problem statement. 

% % ISS OEs (from https://in-the-sky.org/spacecraft_elements.php?id=25544)  

% Rotating and fixed earth plot 
\begin{figure}[H]
	\centering 
	\includegraphics*[width=0.7\textwidth]{1.a Rotating and Non-Rotating Earth.pdf}
	\label{fig:1.a}
\end{figure}


% ------------------------- % 
\subsection*{B}

% Rotating and fixed earth plot 
\begin{figure}[H]
	\centering 
	\includegraphics*[width=0.7\textwidth]{1.b Rotating and Non-Rotating Earth.pdf}
	\label{fig:1.a}
\end{figure}


% ------------------------- % 
\subsection*{C}

% Rotating and fixed earth plot 
\begin{figure}[H]
	\centering 
	\includegraphics*[width=0.7\textwidth]{1.c Fixed and Rotating Earth with Secular Precession Prograde.pdf}
	\label{fig:1.a}
\end{figure}

To derive a formula for predicting the westward drift, first calculate the Draconitic period: 

% Td = 2*pi / (wdot + Mdot); 
% Dn = 2*pi / ( w_E - Odot ); 

% u0 = w0 + M0; 

% u = u0 + udot * T + 2*pi/udot; 

% dlambda = (-w_E - Odot) * Td * 180/pi; 

% pred = lla_rot_c2(end,2) - lla_rot_c2(1,2); 

\begin{equation}
	T_d = \frac{2 \pi}{\dot{\omega} + \dot{M}}
\end{equation}

Then calculate the rate of secular variation for the ascending node due to J2: 

% Odot = -3/2 * n * J2 * ( R_E / norm(rv0(1:3)) )^2 * 1/( 1-e0^2 )^2 * cos(i0); 
\begin{equation}
	\dot{\Omega} = - \frac{3}{2} \bar{n} J_2 \big( \frac{a_e}{ \bar{a} } \big) ^2 \frac{1}{(1-\bar{e}^2)^2} cos \bar{I}
\end{equation}

$\bar{a}$, $\bar{e}$, and $\bar{I}$ can be taken from the initial ISS orbit parameters, and $\bar{n}$ is the mean motion of the orbit: 

% sqrt(mu_E_km3/a0^3)
\begin{equation}
	\bar{n} = \sqrt{\frac{\mu}{\bar{a}^3}}
\end{equation}

Finally, the change in lambda was calculated as: 

\begin{equation}
	\Delta \lambda = -(w_E - \dot{\Omega}) T_d 
\end{equation}

where $w_E$ is the rotation rate of the Earth. $\Delta \lambda$ came out to \textbf{-22.870$^\circ$}. 

The calculated change in drift of the ascending node was taken from the difference between the initial longitude and the final longitude after one revolution, which came out to be \textbf{-22.790$^\circ$}. The calculated drift was within 0.1$^\circ$ of the analytical prediction.  



% \newpage 
% ================================================================ % 
\section*{Problem 2: Orbit Design}

For a sun-synchronous orbit, the $\Omega$ nodal rate needs to match the average rate of the Sun's motion projected onto the Earth's equator: 

\begin{equation}
	\frac{d \Omega}{dt} = \dot{\bar{\Omega}} = \frac{360 ^\circ}{365.242 \; days/year} = 0.9856 \; ^\circ / day 
\label{eq:SSO_dOmega}
\end{equation}




% ------------------------- % 
\subsection*{B}


% ================================================================ % 
% Problem 3 

\section*{Problem 3}

% ------------------------- % 
\subsection*{A}



% ------------------------- % 
\subsection*{B}


% ------------------------- % 
\subsection*{C}



% ================================================================ % 
% Problem 4 

\section*{Problem 4}

% ------------------------- % 
\subsection*{A}



% ------------------------- % 
\subsection*{B}


% \newpage
% ================================================================ % 
\section*{Appendix} 

\subsection*{MATLAB code} 

\begin{lstlisting}

    %% HW 4 
    % Junette Hsin 
    

    
	
\end{lstlisting}





% ================================================================ % 

% \bibliography{sample}

\end{document}
