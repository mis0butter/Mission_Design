\documentclass[conf]{new-aiaa}
%\documentclass[journal]{new-aiaa} for journal papers
\usepackage[utf8]{inputenc}

\usepackage{graphicx}
\usepackage{amsmath}
\usepackage{commath}
\usepackage[version=4]{mhchem}
\usepackage{siunitx}
\usepackage{longtable,tabularx}
\usepackage{float}
\usepackage{listings}
\usepackage{pdfpages}
\usepackage{url}
\usepackage{color} %red, green, blue, yellow, cyan, magenta, black, white
\definecolor{mygreen}{RGB}{28,172,0} % color values Red, Green, Blue
\definecolor{mylilas}{RGB}{170,55,241}
\setlength\LTleft{0pt} 

\lstset{language=Matlab,%
	basicstyle=\footnotesize,
	breaklines=true,%
	morekeywords={matlab2tikz},
	keywordstyle=\color{blue},%
	morekeywords=[2]{1}, keywordstyle=[2]{\color{black}},
	identifierstyle=\color{black},%
	stringstyle=\color{mylilas},
	commentstyle=\color{mygreen},%
	showstringspaces=false,%without this there will be a symbol in the places where there is a space
	numbers=left,%
	numberstyle={\tiny \color{black}},% size of the numbers
	numbersep=9pt, % this defines how far the numbers are from the text
	emph=[1]{for,end,break},emphstyle=[1]\color{red}, %some words to emphasise
	%emph=[2]{word1,word2}, emphstyle=[2]{style},    
}

% ================================================================ % 
\title{ASE387P.2 Mission Analysis and Design \\ Homework 4: Orbit Design}

\author{Junette Hsin}
\affil{Masters Student, Aerospace Engineering and Engineering Mechanics, University of Texas, Austin, TX 78712}

\begin{document}

\maketitle

% \begin{abstract}

	% Theory and algorithms 

% \end{abstract}

% \newpage 
% ================================================================ % 
\section*{Problem 1: ISS Ground Tracks}



% ------------------------- % 
\subsection*{A}

The following parameters were used to initialize the ISS orbit, which were taken from \url{https://in-the-sky.org/spacecraft_elements.php?id=25544} on April 19: 

% e0 = 0.00049; 
% i0 = 51.6427 * pi/180; 
% w0 = 40.1116 * pi/180; 
% O0 = 0; 
% M0 = 70.88 * pi/180; 
\begin{itemize}
	\item $a$ = 6794.588 km 
	\item $e$ = 0.00049 
	\item $i$ = 51.627 deg 
	\item $\omega$ = 40.1116 deg 
	\item $\Omega$ = 0 deg 
	\item $M$ = 70.88 
\end{itemize}

The right ascension of the ascending node, $\Omega$, was set to 0 as stated in the problem statement. 

% % ISS OEs (from https://in-the-sky.org/spacecraft_elements.php?id=25544)  

% Rotating and fixed earth plot 
\begin{figure}[H]
	\centering 
	\includegraphics*[width=0.7\textwidth]{1.a Rotating and Non-Rotating Earth.pdf}
	\label{fig:1.a}
\end{figure}


% ------------------------- % 
\subsection*{B}

% Rotating and fixed earth plot 
\begin{figure}[H]
	\centering 
	\includegraphics*[width=0.7\textwidth]{1.b Rotating and Non-Rotating Earth.pdf}
	\label{fig:1.a}
\end{figure}


% ------------------------- % 
\subsection*{C}

% Rotating and fixed earth plot 
\begin{figure}[H]
	\centering 
	\includegraphics*[width=0.7\textwidth]{1.c Fixed and Rotating Earth with Secular Precession Prograde.pdf}
	\label{fig:1.a}
\end{figure}

To derive a formula for predicting the westward drift, first calculate the Draconitic period: 

% Td = 2*pi / (wdot + Mdot); 
% Dn = 2*pi / ( w_E - Odot ); 

% u0 = w0 + M0; 

% u = u0 + udot * T + 2*pi/udot; 

% dlambda = (-w_E - Odot) * Td * 180/pi; 

% pred = lla_rot_c2(end,2) - lla_rot_c2(1,2); 

\begin{equation}
	T_d = \frac{2 \pi}{\dot{\omega} + \dot{M}}
\end{equation}

Then calculate the rate of secular variation for the ascending node due to J2: 

% Odot = -3/2 * n * J2 * ( R_E / norm(rv0(1:3)) )^2 * 1/( 1-e0^2 )^2 * cos(i0); 
\begin{equation}
	\dot{\Omega} = - \frac{3}{2} \bar{n} J_2 \big( \frac{a_e}{ \bar{a} } \big) ^2 \frac{1}{(1-\bar{e}^2)^2} cos \bar{I}
	\label{eq:Odot}
\end{equation}

$\bar{a}$, $\bar{e}$, and $\bar{I}$ can be taken from the initial ISS orbit parameters, and $\bar{n}$ is the mean motion of the orbit: 

% sqrt(mu_E_km3/a0^3)
\begin{equation}
	\bar{n} = \sqrt{\frac{\mu}{\bar{a}^3}}
\end{equation}

Finally, the change in lambda was calculated as: 

\begin{equation}
	\Delta \lambda = -(w_E - \dot{\Omega}) T_d 
	\label{eq:dlambda}
\end{equation}

where $w_E$ is the rotation rate of the Earth. $\Delta \lambda$ came out to \textbf{-22.870$^\circ$}. 

The calculated change in drift of the ascending node was taken from the difference between the initial longitude and the final longitude after one revolution, which came out to be \textbf{-22.790$^\circ$}. The calculated drift was within 0.1$^\circ$ of the analytical prediction.  



% \newpage 
% ================================================================ % 
\section*{Problem 2: Orbit Design}

% In orbital mechanics, a frozen orbit is an orbit for an artificial satellite in which natural drifting due to the central body's shape has been minimized by careful selection of the orbital parameters.[clarification needed] Typically, this is an orbit in which, over a long period of time, the satellite's altitude remains constant at the same point in each orbit.[1] Changes in the inclination, position of the lowest point of the orbit, and eccentricity have been minimized by choosing initial values so that their perturbations cancel out.[2], which results in a long-term stable orbit that minimizes the use of station-keeping propellant.

Continuing from Equation \ref{eq:dlambda}, after $m$ revolutions, 

\begin{equation}
	m \Delta \lambda = -(w_E - \dot{\Omega}) T_d m 
\end{equation}

A condition for ground track repeat is that $m \Delta \lambda $ needs to be a multiple (let's say $k$) of $2 \pi$, or 

\begin{equation}
	k = \frac{m \Delta \lambda}{2 \pi} = \frac{m ( w_E - \dot{\bar{\Omega}} ) T_d}{ 2 \pi } = m \frac{T_d}{D_n}
\end{equation}

For a ground track repeat orbit, the conditions for which the sub-satellite point repeatedly passes a geographical location ($\phi$, $\lambda$) at regular intervals are given by: 

\begin{equation}
	(\omega_e - \dot{\bar{\Omega}}) D_n = k 2 \pi 
\end{equation}

\begin{equation}
	( \dot{\bar{\omega}} + \dot{\bar{M}} ) T_d = m 2 \pi 
\end{equation}

As for frozen orbits, the largest perturbation on gravitational acceleration is due to J2, which is on the order of 10$^{-3}$. The next-largest perturbation is J3, which is on the order of 10$^-6$. Equation \ref{eq:Odot} gives the scalar secular variation on $\dot{\bar{\Omega}}$ due to J2 which grows indefinitely. Equations \ref{eq:wdot} and \ref{eq:Mdot} give the secular variations due to J2 for $\dot{\bar{\omega}}$ and $\dot{\bar{M}}$: 

\begin{equation}
	\dot{\bar{\omega}} = - \frac{3}{4} \bar{n} J_2 \big( \frac{a_e}{\bar{a}} \big)^2 \frac{1}{( 1 - \bar{e}^2 )^2} (1 - 5 cos^2 \bar{I})
	\label{eq:wdot}
\end{equation}

\begin{equation}
	\dot{\bar{M}} = \bar{n} \Bigg[ 1 - \frac{3}{4} \big( \frac{a_e}{\bar{a}} \big)^2 J_2 \frac{1}{( 1 - \bar{e}^2 )^{\frac{3}{2}}} (1 - 3 cos^2 \bar{I}) \Bigg]
	\label{eq:Mdot}
\end{equation}

J2 also leads to a short-period perturbation on all elements, but only semi-major axis is of interest: 

\begin{equation}
	\Delta a _{SP} (t) = \bar{a} J_2 \big( \frac{a_e}{\bar{a}} \big)^2 \Bigg[ \Big( 1 - \frac{3}{2} sin^2 \bar{I} \Big) \Big( ( \frac{\bar{a}}{r} )^3 - \frac{1}{( 1 - \bar{e}^2 )^{\frac{3}{2}}} \Big) + \frac{3}{2} \big( \frac{\bar{a}}{r} \big)^3 sin^2 \bar{I} cos 2 (\bar{w} + \bar{f}) \Bigg]
\end{equation}

J3 exerts a long-period perturbation on all elements other than the semi-major axis, but only the eccentricity and perigee are of interest: 

\begin{equation}
	\Delta e_{LP}(t) = - \frac{1}{2} \frac{J_3}{J_2} \frac{a_e}{\bar{a}} sin \bar{I} sin \bar{w} (t) 
	\label{eq:Delta_e_LP}
\end{equation}

\begin{equation}
	\bar{e} \Delta \omega _{LP} (t) = - \frac{1}{2} \frac{J_3}{J_2} \frac{a_e}{\bar{a}} \frac{1}{( 1 - \bar{e}^2 )} sin \bar{I} \; cos \bar{w} (t) 
	\label{eq:Delta_w_LP}
\end{equation}

For frozen orbits, the average variation rates of $e$ and $w$ are set to 0. Thus, the mean argument of perigee should be around 90 $^\circ$. 

\begin{equation}
	w = 90 ^\circ
\end{equation}

An orbit is sun-synchronous when the precession rate equals the mean motion of the Earth around the Sun. the $\Omega$ nodal rate needs to match the average rate of the Sun's motion projected onto the Earth's equator: 

\begin{equation}
	\frac{d \Omega}{dt} = \dot{\bar{\Omega}} = \frac{360 ^\circ}{365.242 \; days/year} = 0.9856 \; ^\circ / day 
\label{eq:SSO_dOmega}
\end{equation}

The angular precession for an Earth orbiting satellite is given by Equation \ref{eq:Odot}. One can reform Equation \ref{eq:Odot} as a formula for inclination: 

\begin{equation}
	\bar{I} = cos^{-1} \Bigg[ -\frac{2}{3} \frac{d \Omega}{dt} \frac{1}{J_2 \bar{n}} \Big( \frac{\bar{a}(1 - \bar{e}^2)}{R_E} \Big)^2 \Bigg]
\end{equation}

The process for determining the mean elements $\bar{a}$, $\bar{e}$, and $\bar{I}$ involves making initial guesses and then minimizing the misclosure rate through an optimization routine: 

\begin{equation}
	\epsilon = m(w_E - \dot{\bar{\Omega}}) - k(\dot{\bar{\omega}} - \dot{\bar{M}})
\end{equation}

Initial computed states for position and velocity will be integrated and iterated until a sun-synchronous, frozen, and repeated ground track orbit is found. 


% ------------------------- % 
\subsection*{B}


% ================================================================ % 
% Problem 3 

\section*{Problem 3}

The calculations for this section were taken from the paper \textbf{Five Special Types of Orbits Around Mars} (2010 Liu, Baoyin, and Ma). A frozen orbit is possible on Mars. Mars' J2 = 1.95545e-3 and J3 = 3.14498e-5. In comparison, the Earth J2 = 1.08263e-3 and J3 = -2.53266e-6. The same positive sign between J2 and J3 must be accounted for to prevent the desired eccentricity from becoming negative. For Earth, $\omega$ is set to 90 $^\circ$. For Mars, $\omega$ must be set to around 270 $^\circ$. 

The average variation rate of $e$ from Equation \ref{eq:Delta_e_LP} is given by: 

\begin{equation}
	\dot{\bar{e}} = \frac{3 \bar{n J_3 R_p^3 sin \bar{I}}}{ 4 a^3 ( 1 - e^2 )^2 } \Big( \frac{5}{2} sin^2\bar{I} - 2 \Big) cos \omega
\end{equation}

And from Equation \ref{eq:Delta_w_LP}, the average variation rate of $\omega$ is given by: 

\begin{equation}
	\dot{\bar{\omega}} = \frac{3 \bar{n} J_2 R_p^2}{2 a^2 (1-e^2)^2} \Bigg[ \Big( 2 - \frac{5}{2} sin^2 \bar{I} \Big) \Big( 1 + \frac{J_3 R_p}{2 J_2 a(1-e^2)} \big( \frac{sin^2 \bar{I} - \bar{e}^2 cos^2 \bar{I}}{sin \bar{I}} \big) \frac{sin \omega}{\bar{e}} \Big) + \frac{3 J_2 R_p^2}{2 \bar{a}^2 ( 1 - e^2 )^2} D \Bigg]
\end{equation}

where 

\begin{equation}
	D = (4 + \frac{7}{12} \bar{e}^2 + 2 \sqrt{1 - e^2} ) - sin^2 \bar{I}( \frac{103}{102} + \frac{3}{8} \bar{e}^2 + \frac{11}{2} \sqrt{1 - \bar{e}^2} ) + sin^4 \bar{I}(\frac{215}{418} - \frac{15}{32} \bar{e}^2 + \frac{15}{4} \sqrt{1 - \bar{e}^2} ) + ... H.O.T. 
\end{equation}

H.O.T. stands for higher order terms than J3. As stated in Problem 2, for frozen orbits, the average variation rates of $e$ and $\omega$ are set to 0. 

Solving the above equations yields the following for $e$ and $\omega$: 

\begin{equation}
	e = \frac{ \frac{J_3 R_p}{ 2 J_2 \bar{a} } sin \bar{I} sin{\omega} }{ 1 - \frac{ 3 J_2 R_p^2 E }{ \bar{a}^2 ( 5 sin^2 \bar{I} - 4 ) } }
\end{equation}

\begin{equation}
	\omega = 270 ^\circ
\end{equation}

% \newpage
% ================================================================ % 
\section*{Appendix} 

\subsection*{MATLAB code} 

\begin{lstlisting}

    %% HW 4 
    % Junette Hsin 
    

    
	
\end{lstlisting}





% ================================================================ % 

% \bibliography{sample}

\end{document}
