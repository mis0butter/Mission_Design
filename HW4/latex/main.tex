\documentclass[conf]{new-aiaa}
%\documentclass[journal]{new-aiaa} for journal papers
\usepackage[utf8]{inputenc}

\usepackage{graphicx}
\usepackage{amsmath}
\usepackage{commath}
\usepackage[version=4]{mhchem}
\usepackage{siunitx}
\usepackage{longtable,tabularx}
\usepackage{float}
\usepackage{listings}
\usepackage{pdfpages}
\usepackage{color} %red, green, blue, yellow, cyan, magenta, black, white
\definecolor{mygreen}{RGB}{28,172,0} % color values Red, Green, Blue
\definecolor{mylilas}{RGB}{170,55,241}
\setlength\LTleft{0pt} 

\lstset{language=Matlab,%
	basicstyle=\footnotesize,
	breaklines=true,%
	morekeywords={matlab2tikz},
	keywordstyle=\color{blue},%
	morekeywords=[2]{1}, keywordstyle=[2]{\color{black}},
	identifierstyle=\color{black},%
	stringstyle=\color{mylilas},
	commentstyle=\color{mygreen},%
	showstringspaces=false,%without this there will be a symbol in the places where there is a space
	numbers=left,%
	numberstyle={\tiny \color{black}},% size of the numbers
	numbersep=9pt, % this defines how far the numbers are from the text
	emph=[1]{for,end,break},emphstyle=[1]\color{red}, %some words to emphasise
	%emph=[2]{word1,word2}, emphstyle=[2]{style},    
}

% ================================================================ % 
\title{ASE387P.2 Mission Analysis and Design \\ Homework 4: Orbit Design}

\author{Junette Hsin}
\affil{Masters Student, Aerospace Engineering and Engineering Mechanics, University of Texas, Austin, TX 78712}

\begin{document}

\maketitle

% \begin{abstract}

	% Theory and algorithms 

% \end{abstract}

% \newpage 
% ================================================================ % 
\section*{Problem 1: ISS Ground Tracks}



% ------------------------- % 
\subsection*{A}


% ------------------------- % 
\subsection*{B}


% ------------------------- % 
\subsection*{C}



% \newpage 
% ================================================================ % 
\section*{Problem 2: Orbit Design}

For a sun-synchronous orbit, the $\Omega$ nodal rate needs to match the average rate of the Sun's motion projected onto the Earth's equator: 

\begin{equation}
	\frac{d \Omega}{dt} = \dot{\bar{\Omega}} = \frac{360 ^\circ}{365.242 \; days/year} = 0.9856 \; ^\circ / day 
\label{eq:SSO_dOmega}
\end{equation}




% ------------------------- % 
\subsection*{B}


% ================================================================ % 
% Problem 3 

\section*{Problem 3}

% ------------------------- % 
\subsection*{A}



% ------------------------- % 
\subsection*{B}


% ------------------------- % 
\subsection*{C}



% ================================================================ % 
% Problem 4 

\section*{Problem 4}

% ------------------------- % 
\subsection*{A}



% ------------------------- % 
\subsection*{B}


% \newpage
% ================================================================ % 
\section*{Appendix} 

\subsection*{MATLAB code} 

\begin{lstlisting}

    %% HW 4 
    % Junette Hsin 
    

    
	
\end{lstlisting}





% ================================================================ % 

% \bibliography{sample}

\end{document}
